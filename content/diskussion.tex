\section{Diskussion}
\label{sec:Diskussion}
Die große Abweichung zwischen gemessenen und errechneten Koeffizienten könnte hier auf die Eichung des Oszilloskops zurückzuführen sein.
Diese Vermutung wird dadurch unterstützt, dass der Abfall der Koeffizienten anähernd mit dem erwarteten Wert von $\frac{1}{n^2}$ bzw. $\frac{1}{n}$ eingetreten ist.
Die Koeffizienten wurden also lediglich in der falschen Größenordung angezeigt.
Dies wird zusätlich noch bekräftigt, da alle Koeffizienten eine Abweichung zwichen $82\%-90\%$ zum errechneten Wert aufweisen.
Die Funktionen lassen sich bereits durch wenige Koeffizienten rekonstruieren. Die Überschwingungen, welche bei der Rechteckfunktion sehr gut erkennbar sind, sind nicht vermeidbar.
Durch mehere Koeffizienten würden diese lediglich geringer werden.
An der Dreiecksfunktion ist Approximation schon sehr genau.
Durch mehrere Koeefizienten würde die Auflösung der Funktion jedoch zunehmen. 
